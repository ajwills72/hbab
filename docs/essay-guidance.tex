% Document format
\documentclass[11pt]{article}
% Paper size and margins
\usepackage[a4paper,left=2.25cm,right=2.25cm,top=2cm,bottom=2.5cm]{geometry}
% Professional-looking tables
\usepackage{booktabs}
\usepackage{threeparttable}
% APA style citations and references
\usepackage{apacite}
% Nice web addresses
\usepackage{url}
% Allow whole sections to be commented out easily
\usepackage{comment}

\begin{document}
\title{How To Build A Brain - Essay Guidance}

\author{Andy J. Wills}
\date{\today}
\maketitle


\section*{Summary}
Over the next few pages, I give quite a lot of detail about what I expect from the essay, and how to approach researching and writing your essay. If you follow this advice fully and successfully, it is likely you will get a good mark.

\begin{flushright}
	\emph{Andy J. Wills}
\end{flushright}

\section*{My notes}



\section*{Where to start?}

\paragraph{From my notes to primary sources} There are notes accompanying these lectures, start by reading the notes. Although your essay must follow APA style, my notes do not - where you see a hyperlink (blue text), you can access more information by clicking on it. That information is sometimes a video, sometimes a wikipedia article, sometimes a journal article. Note, and this is very important, that for the purposes of your assessment, you do not have this level of flexibilty. You must cite evidence in APA format, and everything you cite must be a peer-reviewed primary source. Do not cite wikipedia directly, use wikipedia to get an overview and then track down peer-reviewed primary sources using the reference section of those wikipedia articles. Part of the School's assessment criteria require that you demonstrate the ability to cite primary peer-reviewed sources in APA style - do not throw marks away by ignoring this.  

\paragraph{From initial primary sources onwards} The essay titles are rooted in the lecture material, but you are expected to go further than the lecture material. Start with the relevant peer-reviewed primary sources you've found above, and then use these to find more research. Do this by (a) following up relevant studies in their reference sections, (b) doing a cited reference search on these papers (use Google Scholar for this - the ``cited by'' links), (c) searching using appropriate search terms in Google Scholar.  

You'll likely end up with a large number of papers. So, narrow this down on four criteria:

\paragraph{Citations} If a piece of work has been cited a lot, this means it has had a significant impact on the field (not necessarily a positive impact). For the purposes of this essay, you can safely ignore articles cited fewer than 10 times (unless recent, see below).

\paragraph{Recency} The impact of recently-published work is as yet unknown; it is therefore unwise to use citations to select recently published articles. For the purposes of this essay, consider all articles published on or after 1st January 2017 to be recent. Including some recent work in your essay is important.

\paragraph{Relevance} Will this article help you answer the question set? Use the title and, if necessary, the abstract, to work this out. You'll need to use your own judgement to do this -- this is a key skill and part of the assessment.

\paragraph{Availability} The library's electronic resources are good, but not infinite. If the article you've found is not in the library, try searching more generally on the internet - for example, many authors provide copies of their papers on personal websites. Google Scholar is often an excellent way to find papers - be sure to click on ``all X versions'' link at the bottom right of the reference. You could also email authors to ask for a copy if it's not available in the library.

\section*{How to priortise your reading}

You may not have the time to read all of the articles you now have, and you should certainly prioritise your reading, whether you intend to read them all in the end or not. My advice is that you read reviews first. This will give you a good understanding of what the phenomena are, but you should limit your use of them in your essay - they are not primary sources. Then read more specific papers -- your reference section should primarily be comprised of specific papers, not reviews. 

\paragraph{Reviews} Articles that integrate across several previous papers, written by people not limited to (but sometimes including) the authors. 

\paragraph{Specific papers} Articles that give detail on one specific piece of research done by the authors. An understanding of detail is important to show in your essay. 

\section*{How to write a good essay}

\paragraph{Total time hypothesis} The best predictor of how much you learn, retain, and understand is how much time you spend deliberately studying the material, giving it your full attention during periods of study (turn off all notifications on your phone and computer; do not have instagram etc. open in your web browser). 

\paragraph{Space your learning} Do not attempt to read everything in 1-2 days. Put aside some time every week from now to the deadline to read and write for this essay. Sleep consolidates learning.

\paragraph{Answer the question} It is absolutely critical that you answer the question in full, and that it is obvious that you have done so. Essays not fully addressing the question will receive low marks. 

\paragraph{Quality of understanding} I strongly recommend trying to summarize an article in note form immediately after you have read it, referring back to the article as little as possible whilst you do this. Then check your summary against the article for accuracy. Although your essay must be written independently, there is no law against discussing the topic with your friends and colleagues. Indeed, I strongly encourage doing this, as it will improve your understanding and theirs.

\paragraph{Evidence-based argument} Use it! It is not enough to summarize the conclusions of others, you must provide a sense of the evidence that supports those conclusions, and how the conclusions follow from the evidence.

\paragraph{Quality of evidence} Read what you like on the internet to aid your introduction to this area. However, be aware that anyone can post stuff on the internet, and just because it's there doesn't mean it's true, and it certainly doesn't mean you can use it in your reference section. You reference section should \emph{only} contain peer-reviewed primary sources. Do not cite books or book chapters (although again, if you come across something useful, by all means read it).

\paragraph{Quality of structure} Plan your essay before you write it. This means writing a list of bullet points, getting the structure right, and then fleshing out the structure from there. Before starting to write, look at your structure. Does it follow a logical progression, like a good story? Or does it flit back and forth, returning to essentially the same points several times? If the latter, re-order to minimize this. Essay structure has rules; these rules are basically those set out by the Ancient Greeks, and not much has changed since then, see \citeA{rh85}. There is a good brief summary of this classic work on wikipedia.

\paragraph{Appropriate audience} Although I mark the essay, your audience is \emph{not} me. Your audience is essentially a Stage 2 psychology student. If you're talking about a concept or technical term that you learned in Stage 1 or 2, it is not essential to define it. If it is a concept or term you learned in Stage 4, define it before using it.

\paragraph{Quality of Presentation} Make sure your essay reads well, with well-constructed sentences, and an absence of grammatical problems. Make sure you follow APA style slavishly. It you use acronyms, define them before using them. \emph{Write plainly and clearly}. It is not a goal of scientific writing to use a complex phrase or word where a simple one will do; nor is it the goal to write something that looks like a transcript of a chat with your friend.  Express yourself as clearly, precisely, and simply, as possible. 

\section*{Previous year's essays}

Here are the main things that the weaker essays in previous years had in common; essays with two or more of these problems largely failed to reach a 2i standard: 

\paragraph{Not going beyond the lecture material} It is OK to cover some of the
material I directly taught you in class, but you must go beyond this in a
meaningful way to score well.

\paragraph{Absence of critical evaluation of the presented material}
You are not being asked simply to repeat what I said in class, or to take the conclusions of authors at face value. You are being asked to critically evaluate -- broadly speaking, this means making it clear by what you write that you have thought about this for yourself, and that the answer to the question genuinely comes from your understanding, not a surface-level re-hashing of material you do not understand. Make sure \emph{you} pass the Turing test and convince me a human has written your essay, rather than GPT-3! 

\paragraph{Absence of specific detail} See ``Evidence-based argument'', above. 

\paragraph{Too much general text} Not infrequently, poorer essays
used up about 2 pages at the beginning making very general comments
that do not directly answer the question, and they use up the last
page saying what they had already said more briefly (a conclusion
concludes - it draws together your thesis - it is not just a miniature
form of the preceding essay).

\bibliographystyle{apacite}
\bibliography{masterbib3} {}

\end{document}

%%% Local Variables:
%%% mode: latex
%%% TeX-master: t
%%% End:
